\subsection{Optimization Tricks}

\subsubsection{Bit Hacks}
\begin{itemize}
    \item \( x \& -x \) gives the least significant bit in \( x \).
    \item Use \inlinecode!for (int x = m; x;) { --x &= m; }! to loop over all subset
          masks of \( m \) (except \( m \) itself).
    \item To find the next number after \( x \) with the same number of bits set, use:
          \inlinecode!c = x &-x, r = x + c; (((r^x) >> 2) / c) | r!
    \item For cumulative sums of subsets, use:
          \begin{codesnippet}
    for (int b=0; b < k; i++)
        for (int i=0; i < (1<<k); i++) 
            if (i&1<<b) D[i] += D[i ^ (1<<b)]; 
          \end{codesnippet}
\end{itemize}

\subsubsection{Pragmas}
\begin{itemize}
    \item Use \inlinecode!#pragma GCC optimize ("Ofast")! to enable GCC's
          auto-vectorization and better optimization for floating points (assumes
          associativity and turns off denormals).
    \item Use \inlinecode!#pragma GCC target ("avx,avx2")! to potentially double
          performance of vectorized code, though it may cause crashes on older machines.
          Consider using \inlinecode!#pragma GCC target ("sse4")! for older
          architectures.
    \item Use \inlinecode!#pragma GCC optimize ("trapv")! to kill the program on integer
          overflows (but it may slow down execution).
\end{itemize}